\section{Выводы}




Выполнив вторую лабораторную работу по курсу \enquote{Дискретный анализ}, я смог более детально узнать, как работает $std::map$, т.к. в $STL$ эта структура данных реализована именно на красно-чёрном дереве. Изначально я думал, что всё это организовано очень сложно, но выполнив эту работу, я понял, что здесь всего одна сложность - реализовать правильное восстановление свойств дерева, что делается довольно просто, когда под рукой есть Кормен или другой источник информации. После того, как реализовано восстановление свойств остаётся лишь сделать удобный для пользователя интерфейс.


Изначально, как я писал выше, у меня всё работало в полтора раза медленнее. Я думал, так и должно быть, не может же один студент написать такую структуру данных, которая будет работать быстрее \enquote{STL'ной} версии, над которой трудится огромный пласт специалистов. Однако, когда я убрал выброс исключений я был крайне удивлён и рад тому, что у меня получилось написать более эффективную версию этой структуры!!!


\pagebreak

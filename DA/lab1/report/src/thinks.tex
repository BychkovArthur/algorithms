\section{Выводы}

Выполнив первую лабораторную работу по курсу \enquote{Дискретный анализ}, я узнал, что есть сортировки, работающие за линейное время на произвольных данных. Раньше я знал только о \enquote{сортировке-подсчётом}, которая, конечно, работает за линейное время, но применять ее имеет смысл только в том случае, когда размер входных данных близок к разнице между максимальным и минимальными элементами, т.е. когда количество уникальных значений не велико. Однако, для меня стало открытием, что существуют сортировки, позволяющие сортировать любое количество сколь угодно различных чисел (и не только), главное, чтобы каждый элемент можно было разделить на части.

Данные знания пригодится мне в спортивном программирование, где каждая миллисекунда важна. Я написал небольшой $benchmark$ и выяснил, что реализованная мной $LSD$ сортировка на массиве в $10^8$ элементов работает аж в пять раз быстрее, чем встроенная в $C++$ \enquote{std::sort}, являющаяся сортировкой Хоара с кучей оптимизацей!!! 
\pagebreak

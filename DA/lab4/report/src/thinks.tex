\section{Выводы}

Эта лабораторная работа далась мне тяжелее всего. Всё из-за того, что очень мало информации по таким алгоритмам. Если по Красно-Чёрным деревьям информации уйма, то по алгоритму Бойера-Мура её сильно меньше, а по Апостолико-Джанкарло по сути один источник - Гастфилд. Поэтому, приходилось долго вчитываться и понимать, что имеет в виду автор.

Благодаря данной лабораторной работе я больше узнал про поиск в строках, и прям отлично понял Z-функцию, которая очень полезна из-за её простоты. Алгоритм Бойера-Мура так же крайне понятен, однако, писать его куда тяжелее, чем Z-функцию. Однако, c Апостолико-Джанкарло чуть хуже из-за его скудного описания в Гастфилде, без единого примера. А так же не понятен момент, зачем вообще нужен Апостолико-Джанкарло, если единственное его преимущество над Бойером-Муром - простота доказательства линейности? Из-за этой простоты мы жертвует $O(|Text|)$ дополнительной памяти, в то время, как Бойер-Мур использует всего $O(|Pattern|)$ памяти! При том, есть та же Z-функция, в которой смысловой нагрузки - 3 строки, её линейность - очевидна, и использует она те же $O(|Text| + |Pattern|)$ памяти, что и Апостолико-Джанкарло.

\pagebreak

\section{Тест производительности}


Тест производительности представляет из себя следующее: поиск строки в тексте. Алфавит как и в задании - числа в диапазоне от $0$ до $2^{32} - 1$. Тестирование производится на тексте длиной 1 миллион символов. Длина паттерна варьируется, я ее задаю запуском $python$ скрипта, указывая левую и правую границы длины.

\begin{alltt}
> g++ benchmark.cpp
> python3 test-gen.py 3 6
> ./a.out
Apostolico-Giancarlo 32ms
std::search 16ms
> ./a.out
Apostolico-Giancarlo 32ms
std::search 16ms
> diff benchmark_out_ag.txt benchmark_out_find.txt
> python3 test-gen.py 1000 2000
> ./a.out
Apostolico-Giancarlo 20ms
std::search 16ms
> diff benchmark_out_ag.txt benchmark_out_find.txt
> ./a.out
Apostolico-Giancarlo 21ms
std::search 17ms
> ./a.out
Apostolico-Giancarlo 19ms
std::search 16ms
> diff benchmark_out_ag.txt benchmark_out_find.txt
> python3 test-gen.py 100 200
> ./a.out
Apostolico-Giancarlo 23ms
std::search 16ms
> diff benchmark_out_ag.txt benchmark_out_find.txt
> python3 test-gen.py 300 600
> ./a.out
Apostolico-Giancarlo 21ms
std::search 16ms
> ./a.out
Apostolico-Giancarlo 21ms
std::search 17ms
> ./a.out
Apostolico-Giancarlo 15ms
std::search 16ms
\end{alltt}

Но, если урезать мощность алфавита для 3-х символов:

\begin{alltt}
> g++ benchmark.cpp
> python3 test-gen.py 1000 2000
> ./a.out
Apostolico-Giancarlo 16ms
std::search 28ms
> python3 test-gen.py 3 6
> ./a.out
Apostolico-Giancarlo 55ms
std::search 40ms
> python3 test-gen.py 100 200
> ./a.out
Apostolico-Giancarlo 27ms
std::search 28ms
\end{alltt}

Как видно, на случайных данных данный алгоритм может быть хуже наивного. Моё предположение, почему так заключается в том, что из-за M-массива, к которому мы постоянно обращаемся нарушается локальность кэша, что делает долгим выполнение некоторых операций.
Однако, когда происходит много совпадений: маленький алфавит или же реальный текст, в котором может быть много слов похожих на образец, алгоритм довольно эффективен.

\pagebreak


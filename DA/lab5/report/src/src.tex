\section{Описание}

Суффиксный массив для строки $S[1...n]$ - массив целых чисел от $1$ до $n$, такой, что $i$ суффикс $S[i...n]$ исходной строки $s$ лексикографически упорядочен. Т.е. до него стоят суффиксы, которые лексикографически меньше, а после - лексикографически больше. 

Для построения такого массива воспользуемся алгоритмом, который работает за $O(n * log(n))$. Сам алгоритм таков:

\begin{enumerate} 
  \item  Добавим в конец строки сентинел. Причем, сентинел должен быть лексикографически меньше любого символа алфавита, используемого в строке. В $C++$ для $std::string$ гарантируется, что последний символ имеет код $0$. Его можно использовать в качестве сентинела.
  \item  Теперь мысленно дополним каждый суффикс до длины, равной степени двойки. Причем, мы дополним его циклическим сдвигом нашей строки. Например, если у нас строка вместе с сентинелом выглядит так: $abcd\$$, где $\$$ - сентинел, то для суффикса $cd\$$ мы получим строку длины $8$: $cd\$abcd\$$. Очевидно, что новые строки лексикографически останутся в том же порядке.
  \item Теперь, будем строить эти суффиксы поэтапно. Сначала для каждого суффикса возьмем префикс длины $1$. После длины $2, 4, ..., 2 ^ {log(n)}$.
    \begin{enumerate}
        \item Для длины $1$ все очевидно. Просто отсортируем все буквы строки сортировкой подсчетом. Назовем этот массив $p$.
        \item Теперь построим вспомогательный массив - $eq$. Это классы равенства. Если $p[i] = p[i - 1]$, то $eq[i] = eq[i - 1]$, иначе - $eq[i] = eq[i - 1] + 1$. $eq[0] = 0$ - начальное условие.
        \item Предподсчет выше - база. Теперь будет делать переход от $k$ к $2 * k$.
        \begin{enumerate}
            \item На новом шаге мы должны получить массив $p$, в котором будут префиксы длины $2 ^ {log(k) + 1}$ всех суффиксов, а так же, массив $eq$ для нового массива $p$.
            \item Первым делом строим вспомогательный массив $p'$, где $p'[i] = (i - 2 ^ {log(k)}) \% (n + 1)$.
            \item Массив $p'$ содержит индексы начала всех циклических префиксов, длины $2 ^ {log(k) + 1}$, суффиксов строки $s$ с дописанной к ней сентинелом.
            \item Массив $p'$ отсортирован по правой половине циклического префикса $p'[i]$. Нам осталось отсортировать по лейвой половине. Здесь можно привести такую аналогию: У нас есть набор двузначных чисел, который отсортированы по разряду единиц (десятки могут быть не упорядочены). Мы хотим отсортировать этот набор по десяткам, сохранив сортировку по единицам, чтобы получить полностью отсортированный набор двузначных чисел.
            \item Делаем сортировку подсчетом, где ключом будет являться $eq[p'[i]]$, где $eq$ - массив классов равенства с предыдущего шага. После этого шага мы получим верно отсортированный массив $p$ суффиксов для шага $k + 1$. 
            \item Осталось пересчитать массив $eq$ для шага $k + 1$. Для этого просто идем по массиву $p$ (который для шага $k + 1$) и лексикографически сравниваем пары $\{eq[p[i]], eq[(p[i] + 2^{log(k)}) \% (n + 1)]\}$.
        \end{enumerate}
    \end{enumerate}
\end{enumerate}

Когда мы получили суффиксный массив надо научиться искать в нем паттерн.
Для этого можно реализовать наивный алгоритм с $бинарным поиском$, описанный в \cite{search-in-suffix-tree-itmo}, который будет работать за $O((log(n) + k) * |p|)$, где $k$ - количество вхождений паттерна в текст.

Но, этот алгоритм можно значительно ускорить, используя $lcp$. Это ускорение описано в \cite{Gusfield}. Для построения $lcp$ для соседних (в суффиксном массиве) суффиксов можно использовать алгоритм Касаи, Аримуры, Арикавы, Ли, Парка \cite{kasai-algorithm-itmo}. Чтобы получить $lcp$ для любой пары суффиксов, над массивом полученным в алгоритме Касаи можно построить Дерево Отрезков или любую другую структуру, которая решает $RMQ$. Такой поиск (в случае с деревом отрезков) будет работать со сложностью $O(log^2(n) + |p| + k)$. Если же вместо Дерева Отрезков использовать структуру, которая позволит получать $RMQ$ за $O(1)$, то можно достичь сложности поиска $O(log(n) + |p| + k)$.

\pagebreak

\section{Исходный код}

\begin{lstlisting}[language=C++]
#include <algorithm>
#include <cassert>
#include <cstddef>
#include <iostream>
#include <ostream>
#include <string>
#include <vector>
#include <numeric>

template <typename T>
std::ostream& operator<<(std::ostream& os, const std::vector<T> vct) {
    for (std::size_t i = 0; i < vct.size(); ++i) {
        os << vct[i];
        if (i != vct.size() - 1) {
            os << ", ";
        }
    }
    return os;
}

class SegmentTree {
private:
    std::size_t n_;
    std::vector<std::size_t> data_;

    void build(
        std::size_t l,
        std::size_t r,
        std::size_t id,
        const std::vector<std::size_t>& vct
    )
    {
        if (l == r) {
            data_[id] = vct[l];
            return;
        }
        std::size_t m = (l + r) / 2;
        std::size_t lid = id * 2 + 1;
        std::size_t rid = id * 2 + 2;

        build(l, m, lid, vct);
        build(m + 1, r, rid, vct);

        data_[id] = std::min(data_[lid], data_[rid]);
    }

    std::size_t get(
        std::size_t ql,
        std::size_t qr,
        std::size_t l,
        std::size_t r,
        std::size_t id
    )
    {
        if (ql <= l && r <= qr) {
            return data_[id];
        }

        std::size_t m = (l + r) / 2;
        std::size_t lid = id * 2 + 1;
        std::size_t rid = id * 2 + 2;

        if (qr <= m) {
            return get(ql, qr, l, m, lid);
        }
        if (ql > m) {
            return get(ql, qr, m + 1, r, rid);
        }
        return std::min(get(ql, qr, l, m, lid), get(ql, qr, m + 1, r, rid));
    }

public:

    SegmentTree() : n_(0), data_() {}

    SegmentTree(const std::vector<std::size_t>& vct) : n_(vct.size()), data_(n_ * 4, 0) {
        build(0, n_ - 1, 0, vct);
    }

    void build(const std::vector<std::size_t>& vct) {
        n_ = vct.size();
        data_.resize(n_ * 4, 0);
        build(0, n_ - 1, 0, vct);
    }

    std::size_t get(
        std::size_t ql,
        std::size_t qr
    )
    {
        return get(ql, qr, 0, n_ - 1, 0);
    }

};


class SuffixArray {
private:
    const std::string& text_;
    std::size_t n_;
    std::vector<std::size_t> array_;
    std::vector<std::size_t> neighbor_lcp_;
    SegmentTree st_;


    static constexpr std::size_t different_chars = 256;

    inline bool is_sorted(const std::vector<std::size_t>& eq) const {
        /* Сортировку суффиксов можно завершать тогда,
        * когда в массиве eq все числа [0 ... n - 1]
        */
        return (n_ * (n_ - 1) / 2) == std::reduce(eq.begin(), eq.end());
    }

    void build() {
        std::size_t sz = 1;
        std::vector<std::size_t> count_sort(std::max(different_chars, n_), 0);
        std::vector<std::size_t> eq(n_, 0);
        std::vector<std::size_t> eq_buffer(n_, 0);
        std::vector<std::size_t> array_buffer(n_);
        
        /*Первоначальная сортировка подсчетом*/
        for (std::size_t i = 0; i < n_; ++i) {
            ++count_sort[text_[i]];
        }
        for (std::size_t i = 1; i < std::max(different_chars, n_); ++i) {
            count_sort[i] += count_sort[i - 1];
        }
        for (ssize_t i = n_ - 1; i >= 0; --i) {
            array_[--count_sort[text_[i]]] = i;
        }
        /*Первоначальная сортировка подсчетом*/


        /*Первоначальный подсчет eq*/
        for (std::size_t i = 1; i < n_; ++i) {
            if (text_[array_[i]] != text_[array_[i - 1]]) {
                eq[array_[i]] = eq[array_[i - 1]] + 1;
            } else {
                eq[array_[i]] = eq[array_[i - 1]];
            }
        }
        /*Первоначальный подсчет eq*/

        while (!is_sorted(eq)) {
            std::fill(count_sort.begin(), count_sort.end(), 0);
            eq_buffer = eq;
            
            /* Подсчет p'*/
            for (std::size_t i = 0; i < n_; ++i) {
                array_buffer[i] = (array_[i] + n_ - sz) % n_;
            }
            /* Подсчет p'*/

            /* Подсчет p''*/
            for (std::size_t i = 0; i < n_; ++i) {
                ++count_sort[eq[array_buffer[i]]];
            }
            for (std::size_t i = 1; i < n_; ++i) {
                count_sort[i] += count_sort[i - 1];
            }
            for (ssize_t i = n_ - 1; i >= 0; --i) {
                array_[--count_sort[eq[array_buffer[i]]]] = array_buffer[i];
            }
            /* Подсчет p''*/
            
            /* Подсчет eq*/
            eq[0] = 0;
            for (std::size_t i = 1; i < n_; ++i) {
                if (
                    eq_buffer[array_[i - 1]] < eq_buffer[array_[i]] ||
                    eq_buffer[array_[i - 1]] == eq_buffer[array_[i]] && eq_buffer[(array_[i - 1] + sz) % n_] < eq_buffer[(array_[i] + sz) % n_]
                ) {
                    eq[array_[i]] = eq[array_[i - 1]] + 1;
                } else {
                    eq[array_[i]] = eq[array_[i - 1]];
                }
            }
            /* Подсчет eq*/

            sz <<= 1;
        }
    }

    bool is_suffix_lower(std::size_t suffix_id, const std::string& pattern) {
        for (std::size_t i = 0; i < pattern.size() ;++i) {
            if (i > 15) break;
            if (text_[suffix_id + i] > pattern[i]) {
                return false;
            } 
            if (text_[suffix_id + i] < pattern[i]) {
                return true;
            }
        }
        return false;
    }

    bool is_suffix_starts_with(std::size_t suffix_id, const std::string& start) {
        for (std::size_t i = 0; i < start.size(); ++i) {
            if (text_[suffix_id + i] != start[i]) return false;
        }
        return true;
    }

    void kasai() {
        std::vector<std::size_t> lcp(n_ - 1, 0);
        std::vector<std::size_t> array_inverse(n_, 0);

        for (std::size_t i = 0; i < n_; ++i) {
            array_inverse[array_[i]] = i;
        }

        std::size_t k = 0;

        for (std::size_t i = 0; i < n_ - 1; ++i) {


            if (k > 0) {
                --k;
            }
            std::size_t j = array_[array_inverse[i] - 1];

            assert(j >= 0);
            assert(j < n_);

            while (text_[j + k] == text_[i + k]) ++k;

            assert(array_inverse[i] - 1 >= 0);
            assert(array_inverse[i] - 1 < n_ - 1);

            lcp[array_inverse[i] - 1] = k;
        }
        neighbor_lcp_ = lcp;
    }

public:

    SuffixArray(const std::string& text) : text_(text), n_(text_.size() + 1), array_(n_, -1), neighbor_lcp_(n_ - 1), st_() {
        build();
        kasai();
        st_.build(neighbor_lcp_);
    }


    const std::vector<std::size_t>& get_suffix_array() {
        return array_;
    }

    std::vector<std::size_t> find_all(const std::string& pattern) {
        /* O(|si|logn_ + k|si|), где k - количество вхождений */
        std::vector<std::size_t> res;
        if (is_suffix_lower(array_[n_ - 1], pattern)) {
            return res;
        }
        std::size_t l = 0, r = n_ - 1;

        while (l < r - 1) {
            std::size_t m = (l + r) / 2;
            if (is_suffix_lower(array_[m], pattern)) {
                l = m;
            } else {
                r = m;
            }
        }
        for (std::size_t i = r; i < n_; ++i) {
            if (is_suffix_starts_with(array_[i], pattern)) {
                res.push_back(array_[i] + 1);
            } else {
                break;
            }
        }
        std::sort(res.begin(), res.end());

        return res;
    }



    std::vector<std::size_t> find_all_fast(const std::string& pattern) {
        /* O(|si|logn_ + k), где k - количество вхождений */

        std::vector<std::size_t> res;
        if (is_suffix_lower(array_[n_ - 1], pattern)) {
            return res;
        }
        std::size_t l = 0, r = n_ - 1;

        while (l < r - 1) {
            std::size_t m = (l + r) / 2;
            if (is_suffix_lower(array_[m], pattern)) {
                l = m;
            } else {
                r = m;
            }
        }

        if (!is_suffix_starts_with(array_[r], pattern)) {
            return res;
        }
        res.push_back(array_[r] + 1);
        
        for (std::size_t i = r + 1; i < n_; ++i) {

            /*Через алгоритм Касаи... я посчитал lcp для соседей.*/
            if (neighbor_lcp_[i - 1] >= pattern.size()) {
                res.push_back(array_[i] + 1);
            } else {
                break;
            }
        }
        std::sort(res.begin(), res.end());

        return res;
    }

    std::size_t lcp_with_suffix_naive(std::size_t suffix_ind, const std::string& pattern) {
        for (std::size_t i = 0; i < pattern.size(); ++i) {
            if (text_[array_[suffix_ind] + i] != pattern[i]) return i;
        }
        return pattern.size();
    }

    std::size_t common_part_from(std::size_t start, std::size_t suffix_ind, const std::string& pattern) {
        std::size_t k = 0;
        while (start + k < pattern.size() && text_[array_[suffix_ind] + start + k] == pattern[start + k]) ++k;
        return k;
    }

    std::vector<std::size_t> find_all_super_fast(const std::string& pattern) {
        /* O(|si| + log^2n_ + k), где k - количество вхождений */

        std::vector<std::size_t> res;
        if (is_suffix_lower(array_[n_ - 1], pattern)) {
            return res;
        }
        std::size_t L = 0, R = n_ - 1;
        std::size_t l = lcp_with_suffix_naive(L, pattern);
        std::size_t r = lcp_with_suffix_naive(R, pattern);


        while (L < R - 1) {
            std::size_t M = (L + R) / 2;
            std::size_t ml = st_.get(L, M - 1);
            std::size_t mr = st_.get(M, R - 1);

            if (l >= r) {
                if (ml > l) {
                    L = M;
                } else if (ml < l) {
                    R = M;
                    r = ml;
                } else {
                    std::size_t k = common_part_from(l, M, pattern);
                    if (l + k == pattern.size()) {
                        R = M;
                        r = pattern.size();
                    } else if (text_[array_[M] + l + k] < pattern[l + k]) {
                        L = M;
                        l = l + k;
                    } else if (text_[array_[M] + l + k] > pattern[l + k]) {
                        R = M;
                        r = l + k;
                    }
                }
            } else {
                if (mr > r) {
                    R = M;
                } else if (mr < r) {
                    L = M;
                    l = mr;
                } else {
                    std::size_t k = common_part_from(r, M, pattern);
                    if (r + k == pattern.size()) {
                        R = M; // Здесь не зеркалим, т.к. я нашел паттерн и я хочу, чтобы справа были >=
                        r = pattern.size();
                    } else if (text_[array_[M] + r + k] < pattern[r + k]) {
                        L = M;
                        l = r + k;
                    } else if (text_[array_[M] + r + k] > pattern[r + k]) {
                        R = M;
                        r = r + k;
                    }
                }
            }
        }

        if (!is_suffix_starts_with(array_[R], pattern)) {
            return res;
        }
        res.push_back(array_[R] + 1);
        
        for (std::size_t i = R + 1; i < n_; ++i) {

            /*Через алгоритм Касаи... я посчитал lcp для соседей.*/
            if (neighbor_lcp_[i - 1] >= pattern.size()) {
                res.push_back(array_[i] + 1);
            } else {
                break;
            }
        }
        std::sort(res.begin(), res.end());

        return res;
    } 
};

int main() {
    std::ios::sync_with_stdio(false);
    std::cin.tie(NULL);

    std::string text, pattern;
    std::cin >> text;
    SuffixArray sa(text);


    std::size_t i = 0;
    while(std::cin >> pattern) {
        ++i;
        const auto& vct = sa.find_all_super_fast(pattern);

        if (vct.empty()) continue;
        std::cout << i << ": " << vct << '\n';
    }

    return 0;
}
\end{lstlisting}

\pagebreak

\section{Консоль}
\begin{alltt}
> g++ suffix-array.cpp
> ./a.out
abaabb
ab
aabb
cd
1: 1, 4
2: 3
\end{alltt}
\pagebreak


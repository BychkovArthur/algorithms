\CWHeader{Лабораторная работа \textnumero 7}

\CWProblem{
Имеется натуральное число n. За один ход с ним можно произвести следующие действия:

\begin{enumerate}
    \item Вычесть единицу
    \item Разделить на два
    \item Разделить на три
\end{enumerate}

При этом стоимость каждой операции – текущее значение n. Стоимость преобразования - суммарная стоимость всех операций в преобразовании. Вам необходимо с помощью последовательностей указанных операций преобразовать число n в единицу таким образом, чтобы стоимость преобразования была наименьшей. Делить можно только нацело.

{\bfseries Форма вывода:} Выведите на первой строке искомую наименьшую стоимость. Во второй строке должна содержаться последовательность операций. Если было произведено деление на 2 или на 3, выведите /2 (или /3). Если же было вычитание, выведите -1. Все операции выводите разделяя пробелом.



}
\pagebreak

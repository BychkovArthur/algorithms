\section{Описание}

Идея решения очень проста:
\begin{enumerate} 
    \item Если мы в 1, то нам ничего делать не нужно, можем вернуть ноль.
    \item Иначе, текущее число больше единицы, значит его нужно уменьшать. Сделать это можно следующими способами:
        \begin{enumerate} 
            \item Вычесть единицу из текущего числа.
            \item Если текущее число четно - то можем разделить его на 2.
            \item Если текущее число кратно 3 - можем разделить его на 3.
        \end{enumerate}
\end{enumerate}

Итого, надо на каждом шаге алгоритма выбирать действие которым мы получим число, которое имеет наимешьную цену. К полученному минимуму надо прибавить текущее число.

Терминальным условием рекурсии будет то, что текущее число равно 1. В этом случае будем возвращать нолью

\pagebreak

\section{Исходный код}

\begin{lstlisting}[language=C++]
#include <cstdint>
#include <vector>
#include <iostream>

std::vector<int64_t> dp;


void solve(int64_t n) {
    if (n == 1) return;

    auto check_memoization = [](int64_t n){
        if (dp[n] == -1) {
            solve(n);
        }
        return dp[n];
    };

    int64_t x1 = check_memoization(n - 1);
    int64_t x2 = n % 2 == 0 ? check_memoization(n / 2) : INT64_MAX;
    int64_t x3 = n % 3 == 0 ? check_memoization(n / 3) : INT64_MAX;

    dp[n] = std::min(std::min(x1, x2), x3) + n;
}

int main() {
    int n;
    std::cin >> n;
    
    std::vector<std::string> operation_types;
    dp.resize(n + 1, -1);
    dp[0] = INT64_MAX;
    dp[1] = 0;

    solve(n);

    std::cout << dp[n] << std::endl;
    while (n != 1) {
        if (dp[n - 1] + n == dp[n]) operation_types.push_back("-1"), n -= 1;
        else if (n % 2 == 0 && dp[n / 2] + n == dp[n]) operation_types.push_back("/2"), n /= 2;
        else if (n % 3 == 0 && dp[n / 3] + n == dp[n]) operation_types.push_back("/3"), n /= 3;
    }

    for (auto it = operation_types.begin(); it != operation_types.end(); ++it) {
        std::cout << *it;
        if (it < operation_types.end() - 1) std::cout << ' ';
    }
}
\end{lstlisting}

\pagebreak

\section{Консоль}
\begin{alltt}
❯ g++ descending.cpp
❯ ./a.out
82
202
-1 /3 /3 /3 /3%  
\end{alltt}
\pagebreak


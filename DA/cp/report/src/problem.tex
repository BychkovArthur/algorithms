\CWHeader{Курсовой проект на тему $BWT, MTF, RLE$}

\CWProblem{
Для BWT в конец строки добавляется символ «\$» для помощи в дальнейшем декодировании. Это может быть любой символ меньший чем «a».
Для $MTF$ используется следующее начальное распределение кодов:
$ \\
\$ \Longrightarrow 0 \\
a \Longrightarrow 1 \\
b \Longrightarrow 2 \\
c \Longrightarrow 3 \\
… \\ 
x \Longrightarrow 24 \\
y \Longrightarrow 25 \\
z \Longrightarrow 26 \\
$
Максимальная длинна текста в тестах $10^5$

{\bfseries Формат ввода: }
Вам будут даны тесты двух типов.
Первый тип:

$compress$

$<text>$

Текст состоит только из малых латинских букв. В ответ на него вам нужно вывести коды, которыми будет закодирован данный текст.

Второй тип:

$decompress$

$<codes>$

Вам даны коды в которые был сжат текст из малых латинских букв, вам нужно его разжать. В RLE коды записываются в порядке (количество, значение)

{\bfseries Формат вывода: }
В ответ на первый тип тестов выведите коды, которыми будет закодирован текст. Каждая пара на отдельной строке.
На второй тип тестов выведите разжатый текст.


}
\pagebreak
